% Options for packages loaded elsewhere
\PassOptionsToPackage{unicode}{hyperref}
\PassOptionsToPackage{hyphens}{url}
%
\documentclass[
  10pt,
  ignorenonframetext,
]{beamer}
\usepackage{pgfpages}
\setbeamertemplate{caption}[numbered]
\setbeamertemplate{caption label separator}{: }
\setbeamercolor{caption name}{fg=normal text.fg}
\beamertemplatenavigationsymbolsempty
% Prevent slide breaks in the middle of a paragraph
\widowpenalties 1 10000
\raggedbottom
\setbeamertemplate{part page}{
  \centering
  \begin{beamercolorbox}[sep=16pt,center]{part title}
    \usebeamerfont{part title}\insertpart\par
  \end{beamercolorbox}
}
\setbeamertemplate{section page}{
  \centering
  \begin{beamercolorbox}[sep=12pt,center]{part title}
    \usebeamerfont{section title}\insertsection\par
  \end{beamercolorbox}
}
\setbeamertemplate{subsection page}{
  \centering
  \begin{beamercolorbox}[sep=8pt,center]{part title}
    \usebeamerfont{subsection title}\insertsubsection\par
  \end{beamercolorbox}
}
\AtBeginPart{
  \frame{\partpage}
}
\AtBeginSection{
  \ifbibliography
  \else
    \frame{\sectionpage}
  \fi
}
\AtBeginSubsection{
  \frame{\subsectionpage}
}

\usepackage{amsmath,amssymb}
\usepackage{lmodern}
\usepackage{iftex}
\ifPDFTeX
  \usepackage[T1]{fontenc}
  \usepackage[utf8]{inputenc}
  \usepackage{textcomp} % provide euro and other symbols
\else % if luatex or xetex
  \usepackage{unicode-math}
  \defaultfontfeatures{Scale=MatchLowercase}
  \defaultfontfeatures[\rmfamily]{Ligatures=TeX,Scale=1}
\fi
\usetheme[]{AAUsimple}
% Use upquote if available, for straight quotes in verbatim environments
\IfFileExists{upquote.sty}{\usepackage{upquote}}{}
\IfFileExists{microtype.sty}{% use microtype if available
  \usepackage[]{microtype}
  \UseMicrotypeSet[protrusion]{basicmath} % disable protrusion for tt fonts
}{}
\makeatletter
\@ifundefined{KOMAClassName}{% if non-KOMA class
  \IfFileExists{parskip.sty}{%
    \usepackage{parskip}
  }{% else
    \setlength{\parindent}{0pt}
    \setlength{\parskip}{6pt plus 2pt minus 1pt}}
}{% if KOMA class
  \KOMAoptions{parskip=half}}
\makeatother
\usepackage{xcolor}
\newif\ifbibliography
\setlength{\emergencystretch}{3em} % prevent overfull lines
\setcounter{secnumdepth}{-\maxdimen} % remove section numbering

\usepackage{color}
\usepackage{fancyvrb}
\newcommand{\VerbBar}{|}
\newcommand{\VERB}{\Verb[commandchars=\\\{\}]}
\DefineVerbatimEnvironment{Highlighting}{Verbatim}{commandchars=\\\{\}}
% Add ',fontsize=\small' for more characters per line
\usepackage{framed}
\definecolor{shadecolor}{RGB}{241,243,245}
\newenvironment{Shaded}{\begin{snugshade}}{\end{snugshade}}
\newcommand{\AlertTok}[1]{\textcolor[rgb]{0.68,0.00,0.00}{#1}}
\newcommand{\AnnotationTok}[1]{\textcolor[rgb]{0.37,0.37,0.37}{#1}}
\newcommand{\AttributeTok}[1]{\textcolor[rgb]{0.40,0.45,0.13}{#1}}
\newcommand{\BaseNTok}[1]{\textcolor[rgb]{0.68,0.00,0.00}{#1}}
\newcommand{\BuiltInTok}[1]{\textcolor[rgb]{0.00,0.23,0.31}{#1}}
\newcommand{\CharTok}[1]{\textcolor[rgb]{0.13,0.47,0.30}{#1}}
\newcommand{\CommentTok}[1]{\textcolor[rgb]{0.37,0.37,0.37}{#1}}
\newcommand{\CommentVarTok}[1]{\textcolor[rgb]{0.37,0.37,0.37}{\textit{#1}}}
\newcommand{\ConstantTok}[1]{\textcolor[rgb]{0.56,0.35,0.01}{#1}}
\newcommand{\ControlFlowTok}[1]{\textcolor[rgb]{0.00,0.23,0.31}{#1}}
\newcommand{\DataTypeTok}[1]{\textcolor[rgb]{0.68,0.00,0.00}{#1}}
\newcommand{\DecValTok}[1]{\textcolor[rgb]{0.68,0.00,0.00}{#1}}
\newcommand{\DocumentationTok}[1]{\textcolor[rgb]{0.37,0.37,0.37}{\textit{#1}}}
\newcommand{\ErrorTok}[1]{\textcolor[rgb]{0.68,0.00,0.00}{#1}}
\newcommand{\ExtensionTok}[1]{\textcolor[rgb]{0.00,0.23,0.31}{#1}}
\newcommand{\FloatTok}[1]{\textcolor[rgb]{0.68,0.00,0.00}{#1}}
\newcommand{\FunctionTok}[1]{\textcolor[rgb]{0.28,0.35,0.67}{#1}}
\newcommand{\ImportTok}[1]{\textcolor[rgb]{0.00,0.46,0.62}{#1}}
\newcommand{\InformationTok}[1]{\textcolor[rgb]{0.37,0.37,0.37}{#1}}
\newcommand{\KeywordTok}[1]{\textcolor[rgb]{0.00,0.23,0.31}{#1}}
\newcommand{\NormalTok}[1]{\textcolor[rgb]{0.00,0.23,0.31}{#1}}
\newcommand{\OperatorTok}[1]{\textcolor[rgb]{0.37,0.37,0.37}{#1}}
\newcommand{\OtherTok}[1]{\textcolor[rgb]{0.00,0.23,0.31}{#1}}
\newcommand{\PreprocessorTok}[1]{\textcolor[rgb]{0.68,0.00,0.00}{#1}}
\newcommand{\RegionMarkerTok}[1]{\textcolor[rgb]{0.00,0.23,0.31}{#1}}
\newcommand{\SpecialCharTok}[1]{\textcolor[rgb]{0.37,0.37,0.37}{#1}}
\newcommand{\SpecialStringTok}[1]{\textcolor[rgb]{0.13,0.47,0.30}{#1}}
\newcommand{\StringTok}[1]{\textcolor[rgb]{0.13,0.47,0.30}{#1}}
\newcommand{\VariableTok}[1]{\textcolor[rgb]{0.07,0.07,0.07}{#1}}
\newcommand{\VerbatimStringTok}[1]{\textcolor[rgb]{0.13,0.47,0.30}{#1}}
\newcommand{\WarningTok}[1]{\textcolor[rgb]{0.37,0.37,0.37}{\textit{#1}}}

\providecommand{\tightlist}{%
  \setlength{\itemsep}{0pt}\setlength{\parskip}{0pt}}\usepackage{longtable,booktabs,array}
\usepackage{calc} % for calculating minipage widths
\usepackage{caption}
% Make caption package work with longtable
\makeatletter
\def\fnum@table{\tablename~\thetable}
\makeatother
\usepackage{graphicx}
\makeatletter
\def\maxwidth{\ifdim\Gin@nat@width>\linewidth\linewidth\else\Gin@nat@width\fi}
\def\maxheight{\ifdim\Gin@nat@height>\textheight\textheight\else\Gin@nat@height\fi}
\makeatother
% Scale images if necessary, so that they will not overflow the page
% margins by default, and it is still possible to overwrite the defaults
% using explicit options in \includegraphics[width, height, ...]{}
\setkeys{Gin}{width=\maxwidth,height=\maxheight,keepaspectratio}
% Set default figure placement to htbp
\makeatletter
\def\fps@figure{htbp}
\makeatother

\def\tightlist{}
\usepackage[utf8]{inputenc}
\usepackage[english]{babel}
\usepackage[T1]{fontenc}
\usepackage{helvet}
\title{Introduktion til R III}
\date{12. maj 2023}
\author{Kristian G. Kjelmann (\href{mailto:kgk@socsci.aau.dk}{{\tt kgk@socsci.aau.dk}})\\
        \&\\
        Rolf L. Lund (\href{mailto:rolfll@socsci.aau.dk}{{\tt rolfll@socsci.aau.dk}})
        }
\institute{Institut for Sociologi og Socialt Arbejde}
\pgfdeclareimage[height=1.5cm]{titlepagelogo}{AAUgraphics/aau_logo_new}
\titlegraphic{
  \pgfuseimage{titlepagelogo}
  }
\begin{document}
  \begin{frame}[plain,noframenumbering]
    \titlepage
  \end{frame}
\makeatletter
\makeatother
\makeatletter
\makeatother
\makeatletter
\@ifpackageloaded{caption}{}{\usepackage{caption}}
\AtBeginDocument{%
\ifdefined\contentsname
  \renewcommand*\contentsname{Table of contents}
\else
  \newcommand\contentsname{Table of contents}
\fi
\ifdefined\listfigurename
  \renewcommand*\listfigurename{List of Figures}
\else
  \newcommand\listfigurename{List of Figures}
\fi
\ifdefined\listtablename
  \renewcommand*\listtablename{List of Tables}
\else
  \newcommand\listtablename{List of Tables}
\fi
\ifdefined\figurename
  \renewcommand*\figurename{Figure}
\else
  \newcommand\figurename{Figure}
\fi
\ifdefined\tablename
  \renewcommand*\tablename{Table}
\else
  \newcommand\tablename{Table}
\fi
}
\@ifpackageloaded{float}{}{\usepackage{float}}
\floatstyle{ruled}
\@ifundefined{c@chapter}{\newfloat{codelisting}{h}{lop}}{\newfloat{codelisting}{h}{lop}[chapter]}
\floatname{codelisting}{Listing}
\newcommand*\listoflistings{\listof{codelisting}{List of Listings}}
\makeatother
\makeatletter
\@ifpackageloaded{caption}{}{\usepackage{caption}}
\@ifpackageloaded{subcaption}{}{\usepackage{subcaption}}
\makeatother
\makeatletter
\@ifpackageloaded{tcolorbox}{}{\usepackage[many]{tcolorbox}}
\makeatother
\makeatletter
\@ifundefined{shadecolor}{\definecolor{shadecolor}{rgb}{.97, .97, .97}}
\makeatother
\makeatletter
\makeatother
\ifLuaTeX
  \usepackage{selnolig}  % disable illegal ligatures
\fi
\IfFileExists{bookmark.sty}{\usepackage{bookmark}}{\usepackage{hyperref}}
\IfFileExists{xurl.sty}{\usepackage{xurl}}{} % add URL line breaks if available
\urlstyle{same} % disable monospaced font for URLs
\hypersetup{
  hidelinks,
  pdfcreator={LaTeX via pandoc}}

\author{}
\date{}

\begin{document}
\ifdefined\Shaded\renewenvironment{Shaded}{\begin{tcolorbox}[borderline west={3pt}{0pt}{shadecolor}, breakable, enhanced, sharp corners, interior hidden, boxrule=0pt, frame hidden]}{\end{tcolorbox}}\fi

\begin{frame}{Dagens program}
\protect\hypertarget{dagens-program}{}
\begin{enumerate}
\item
  Opsamling på datahåndtering
\item
  Eksport af statistiske modeller (med stargazer)
\item
  Opsætning af modeller i tekstpublikationer
\item
  Øvelser
\end{enumerate}
\end{frame}

\begin{frame}{Dagens læringsmål}
\protect\hypertarget{dagens-luxe6ringsmuxe5l}{}
\begin{itemize}
\item
  I kan lave publicérbare modeller med R
\item
  I kan løse forskelligartede datahåndteringsudfordringer i det samme
  datasæt
\end{itemize}
\end{frame}

\begin{frame}{Datahåndteringsbegreber}
\protect\hypertarget{datahuxe5ndteringsbegreber}{}
\textbf{Har I styr på jeres datahåndteringsbegreber?}

\begin{itemize}
\item
  Subsetting
\item
  Filtrering
\item
  Variabelændringer
\item
  Rekodning
\item
  Missingværdier
\end{itemize}
\end{frame}

\begin{frame}[fragile]{Statistiske modeller i R}
\protect\hypertarget{statistiske-modeller-i-r}{}
Der er to dele i at få resultaterne af en statistisk model frem i R:

\begin{enumerate}
\item
  Specificér modellen (fx med en funktion som \texttt{lm()} for lineære
  modeller)
\item
  Få koefficienter og resultater fra modellen (med brug af funktionen
  \texttt{summary()})
\end{enumerate}
\end{frame}

\begin{frame}[fragile]{Statistiske modeller i R - Specificer model}
\protect\hypertarget{statistiske-modeller-i-r---specificer-model}{}
Tre ting krævet for at specificere en model:

\begin{enumerate}
\item
  Et datasæt
\item
  Funktion for typen af model, man vil lave
\item
  Formel, der specificerer det sammenhæng, som man vil modellere
\end{enumerate}

Al datahåndtering i datasæt skal ske \emph{inden} man laver modellen.

Man specificerer en formel med R's formelsyntax, fx
\texttt{y\ \textasciitilde{}\ x1\ +\ x2\ +\ x3}.

Eksempel:

\begin{Shaded}
\begin{Highlighting}[]
\NormalTok{grsp\_model }\OtherTok{\textless{}{-}} \FunctionTok{lm}\NormalTok{(grspnum }\SpecialCharTok{\textasciitilde{}}\NormalTok{ eduyrs }\SpecialCharTok{+}\NormalTok{ wkhtot, }\AttributeTok{data =}\NormalTok{ ess18)}
\end{Highlighting}
\end{Shaded}
\end{frame}

\begin{frame}[fragile]{Statistiske modeller i R - Resultater}
\protect\hypertarget{statistiske-modeller-i-r---resultater}{}
En model i R er blot endnu en type objekt.

For at se resultatet af modellen, skal man derfor spørge R korrekt om
det.

Ved blot at kalde modellen får man begrænsede resultater:

\small

\begin{Shaded}
\begin{Highlighting}[]
\NormalTok{grsp\_model}
\end{Highlighting}
\end{Shaded}

\begin{verbatim}

Call:
lm(formula = grspnum ~ eduyrs + wkhtot, data = ess18)

Coefficients:
(Intercept)       eduyrs       wkhtot  
    -3668.7        891.1        888.4  
\end{verbatim}
\end{frame}

\begin{frame}[fragile]{Statistiske modeller i R - Resultater}
\protect\hypertarget{statistiske-modeller-i-r---resultater-1}{}
Ved at bruge \texttt{summary()} funktionen gives de relevante
resultater:

\tiny

\begin{Shaded}
\begin{Highlighting}[]
\FunctionTok{summary}\NormalTok{(grsp\_model)}
\end{Highlighting}
\end{Shaded}

\begin{verbatim}

Call:
lm(formula = grspnum ~ eduyrs + wkhtot, data = ess18)

Residuals:
    Min      1Q  Median      3Q     Max 
 -64614  -16639  -10324   -1448 2955649 

Coefficients:
            Estimate Std. Error t value Pr(>|t|)  
(Intercept)  -3668.7    20645.3  -0.178   0.8590  
eduyrs         891.1      856.4   1.041   0.2984  
wkhtot         888.4      458.2   1.939   0.0529 .
---
Signif. codes:  0 '***' 0.001 '**' 0.01 '*' 0.05 '.' 0.1 ' ' 1

Residual standard error: 121600 on 707 degrees of freedom
  (575 observations deleted due to missingness)
Multiple R-squared:  0.007698,  Adjusted R-squared:  0.004891 
F-statistic: 2.742 on 2 and 707 DF,  p-value: 0.06511
\end{verbatim}
\end{frame}

\begin{frame}{Statistiske modeller i R - factors}
\protect\hypertarget{statistiske-modeller-i-r---factors}{}
Som standard vil R behandle tekstvariable i statistiske modeller som
\emph{unordered factors}; altså nominalt skalerede variable.

R vil desuden tage kategorien, der kommer først i alfabetisk rækkefølge,
som referencekategori.

Dette kan lede til uhensigtsmæssige resultater. Man bør derfor altid
tage aktiv stilling til, hvordan den kategoriske variabel skal
behandles, inden man laver modellen.
\end{frame}

\begin{frame}[fragile]{Statistiske modeller i R - factors}
\protect\hypertarget{statistiske-modeller-i-r---factors-1}{}
\textbf{Overvej følgende:}

\begin{itemize}
\item
  Skal variablen behandles som \emph{ordinal} eller \emph{nominal}?
  (\texttt{ordered} eller \texttt{unordered})
\item
  Skal variablen behandles som \emph{intervalskaleret}? (konvertér til
  numerisk)
\item
  Hvis variablen skal behandles som nominal, hvilken kategori skal så
  være \emph{referencekategorien}? (kan fx ændres med \texttt{relevel} i
  \texttt{mutate} funktionen)
\end{itemize}

\emph{HUSK}: Der er forskel på hvad variablen \emph{er}, og hvordan vi
behandler den i en model!
\end{frame}

\begin{frame}{Eksport af statistiske modeller}
\protect\hypertarget{eksport-af-statistiske-modeller}{}
\begin{figure}

{\centering \includegraphics[width=0.8\textwidth,height=\textheight]{img/model_egon-drake.png}

}

\end{figure}
\end{frame}



\end{document}
