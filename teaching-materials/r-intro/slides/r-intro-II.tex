% Options for packages loaded elsewhere
\PassOptionsToPackage{unicode}{hyperref}
\PassOptionsToPackage{hyphens}{url}
%
\documentclass[
  10pt,
  ignorenonframetext,
]{beamer}
\usepackage{pgfpages}
\setbeamertemplate{caption}[numbered]
\setbeamertemplate{caption label separator}{: }
\setbeamercolor{caption name}{fg=normal text.fg}
\beamertemplatenavigationsymbolsempty
% Prevent slide breaks in the middle of a paragraph
\widowpenalties 1 10000
\raggedbottom
\setbeamertemplate{part page}{
  \centering
  \begin{beamercolorbox}[sep=16pt,center]{part title}
    \usebeamerfont{part title}\insertpart\par
  \end{beamercolorbox}
}
\setbeamertemplate{section page}{
  \centering
  \begin{beamercolorbox}[sep=12pt,center]{part title}
    \usebeamerfont{section title}\insertsection\par
  \end{beamercolorbox}
}
\setbeamertemplate{subsection page}{
  \centering
  \begin{beamercolorbox}[sep=8pt,center]{part title}
    \usebeamerfont{subsection title}\insertsubsection\par
  \end{beamercolorbox}
}
\AtBeginPart{
  \frame{\partpage}
}
\AtBeginSection{
  \ifbibliography
  \else
    \frame{\sectionpage}
  \fi
}
\AtBeginSubsection{
  \frame{\subsectionpage}
}

\usepackage{amsmath,amssymb}
\usepackage{lmodern}
\usepackage{iftex}
\ifPDFTeX
  \usepackage[T1]{fontenc}
  \usepackage[utf8]{inputenc}
  \usepackage{textcomp} % provide euro and other symbols
\else % if luatex or xetex
  \usepackage{unicode-math}
  \defaultfontfeatures{Scale=MatchLowercase}
  \defaultfontfeatures[\rmfamily]{Ligatures=TeX,Scale=1}
\fi
\usetheme[]{AAUsimple}
% Use upquote if available, for straight quotes in verbatim environments
\IfFileExists{upquote.sty}{\usepackage{upquote}}{}
\IfFileExists{microtype.sty}{% use microtype if available
  \usepackage[]{microtype}
  \UseMicrotypeSet[protrusion]{basicmath} % disable protrusion for tt fonts
}{}
\makeatletter
\@ifundefined{KOMAClassName}{% if non-KOMA class
  \IfFileExists{parskip.sty}{%
    \usepackage{parskip}
  }{% else
    \setlength{\parindent}{0pt}
    \setlength{\parskip}{6pt plus 2pt minus 1pt}}
}{% if KOMA class
  \KOMAoptions{parskip=half}}
\makeatother
\usepackage{xcolor}
\newif\ifbibliography
\setlength{\emergencystretch}{3em} % prevent overfull lines
\setcounter{secnumdepth}{-\maxdimen} % remove section numbering


\providecommand{\tightlist}{%
  \setlength{\itemsep}{0pt}\setlength{\parskip}{0pt}}\usepackage{longtable,booktabs,array}
\usepackage{calc} % for calculating minipage widths
\usepackage{caption}
% Make caption package work with longtable
\makeatletter
\def\fnum@table{\tablename~\thetable}
\makeatother
\usepackage{graphicx}
\makeatletter
\def\maxwidth{\ifdim\Gin@nat@width>\linewidth\linewidth\else\Gin@nat@width\fi}
\def\maxheight{\ifdim\Gin@nat@height>\textheight\textheight\else\Gin@nat@height\fi}
\makeatother
% Scale images if necessary, so that they will not overflow the page
% margins by default, and it is still possible to overwrite the defaults
% using explicit options in \includegraphics[width, height, ...]{}
\setkeys{Gin}{width=\maxwidth,height=\maxheight,keepaspectratio}
% Set default figure placement to htbp
\makeatletter
\def\fps@figure{htbp}
\makeatother

\def\tightlist{}
\usepackage[utf8]{inputenc}
\usepackage[english]{babel}
\usepackage[T1]{fontenc}
\usepackage{helvet}
\title{Introduktion til R II}
\date{7. februar 2023}
\author{Kristian G. Kjelmann (\href{mailto:kgk@socsci.aau.dk}{{\tt kgk@socsci.aau.dk}})\\
        \&\\
        Rolf L. Lund (\href{mailto:rolfll@socsci.aau.dk}{{\tt rolfll@socsci.aau.dk}})
        }
\institute{Institut for Sociologi og Socialt Arbejde}
\pgfdeclareimage[height=1.5cm]{titlepagelogo}{AAUgraphics/aau_logo_new}
\titlegraphic{
  \pgfuseimage{titlepagelogo}
  }
\begin{document}
  \begin{frame}[plain,noframenumbering]
    \titlepage
  \end{frame}
\makeatletter
\makeatother
\makeatletter
\makeatother
\makeatletter
\@ifpackageloaded{caption}{}{\usepackage{caption}}
\AtBeginDocument{%
\ifdefined\contentsname
  \renewcommand*\contentsname{Table of contents}
\else
  \newcommand\contentsname{Table of contents}
\fi
\ifdefined\listfigurename
  \renewcommand*\listfigurename{List of Figures}
\else
  \newcommand\listfigurename{List of Figures}
\fi
\ifdefined\listtablename
  \renewcommand*\listtablename{List of Tables}
\else
  \newcommand\listtablename{List of Tables}
\fi
\ifdefined\figurename
  \renewcommand*\figurename{Figure}
\else
  \newcommand\figurename{Figure}
\fi
\ifdefined\tablename
  \renewcommand*\tablename{Table}
\else
  \newcommand\tablename{Table}
\fi
}
\@ifpackageloaded{float}{}{\usepackage{float}}
\floatstyle{ruled}
\@ifundefined{c@chapter}{\newfloat{codelisting}{h}{lop}}{\newfloat{codelisting}{h}{lop}[chapter]}
\floatname{codelisting}{Listing}
\newcommand*\listoflistings{\listof{codelisting}{List of Listings}}
\makeatother
\makeatletter
\@ifpackageloaded{caption}{}{\usepackage{caption}}
\@ifpackageloaded{subcaption}{}{\usepackage{subcaption}}
\makeatother
\makeatletter
\@ifpackageloaded{tcolorbox}{}{\usepackage[many]{tcolorbox}}
\makeatother
\makeatletter
\@ifundefined{shadecolor}{\definecolor{shadecolor}{rgb}{.97, .97, .97}}
\makeatother
\makeatletter
\makeatother
\ifLuaTeX
  \usepackage{selnolig}  % disable illegal ligatures
\fi
\IfFileExists{bookmark.sty}{\usepackage{bookmark}}{\usepackage{hyperref}}
\IfFileExists{xurl.sty}{\usepackage{xurl}}{} % add URL line breaks if available
\urlstyle{same} % disable monospaced font for URLs
\hypersetup{
  hidelinks,
  pdfcreator={LaTeX via pandoc}}

\author{}
\date{}

\begin{document}
\ifdefined\Shaded\renewenvironment{Shaded}{\begin{tcolorbox}[interior hidden, boxrule=0pt, sharp corners, frame hidden, enhanced, borderline west={3pt}{0pt}{shadecolor}, breakable]}{\end{tcolorbox}}\fi

\begin{frame}{Dagens program}
\protect\hypertarget{dagens-program}{}
\begin{enumerate}
\item
  Hvad er datahåndtering?
\item
  Subsetting i R
\item
  Tilføje nye variable i R
\item
  Rekodning i R
\item
  Håndtering af missingværdier i R
\item
  Tips til god kodepraksis
\end{enumerate}
\end{frame}

\begin{frame}{Dagens læringsmål}
\protect\hypertarget{dagens-luxe6ringsmuxe5l}{}
\begin{itemize}
\item
  I ved hvad datahåndtering indebærer
\item
  I får kendskab til typiske datahåndteringsudfordringer
\item
  I kan foretage relevante datahåndteringsoperationer i R (subsetting,
  nye variable, rekodning, håndtering af missing)
\item
  I kan skrive et R script, der foretager relevant datahåndtering af et
  datasæt
\item
  I kan skrive et R script, som andre kan læse og forstå
\end{itemize}
\end{frame}

\begin{frame}{Hvad er datahåndtering?}
\protect\hypertarget{hvad-er-datahuxe5ndtering}{}
Når man arbejder med data, er man næsten altid nødt til at foretage
visse ændringer i data, før at de er egnet til analyse. Dette kaldes
``datahåndtering''.

\emph{Datahåndtering inkluderer blandt andet:}

\begin{itemize}
\tightlist
\item
  Udvælge specifikke observationer og variable (kaldes også subsetting)
\item
  Danne nye variables
\item
  Rekode værdier
\end{itemize}
\end{frame}

\begin{frame}{Subsetting}
\protect\hypertarget{subsetting}{}
``Subsetting'' henviser generelt til den del af datahåndtering, der har
at gøre med at udvælge de relevante dele af data. Hvad der er relevant
skal altid ses ift. den analyse, som man foretager.\\
\strut \\
\emph{Subsetting inkluderer blandt andet:}

\begin{itemize}
\tightlist
\item
  Begrænse data til specifikke variable/kolonner
\item
  Begrænse data til en bestemt tidsperiode, geografisk område eller
  andet
\item
  Udvælge observationer i data ud fra bestemte betingelser (filtrering)
\end{itemize}

Filtrering er den mest centrale del af subsetting, da det er i dette
arbejde, at man sørger for, at data stemmer overens med den population,
som man undersøger.
\end{frame}

\begin{frame}{Subsetting}
\protect\hypertarget{subsetting-1}{}
\begin{figure}

{\centering \includegraphics[width=0.5\textwidth,height=\textheight]{img/req.jpg}

}

\end{figure}
\end{frame}

\begin{frame}{Brug af betingelser i R}
\protect\hypertarget{brug-af-betingelser-i-r}{}
Subsetting og særligt filtrering er i sidste ende et spørgsmål om at
sætte regler for sine data: hvilke betingelser skal observationer
opfylde for at blive inkluderet i analysen?

Samme logik gør sig gældende i måden, som vi filtrerer data i R, da vi i
R kan spørge, om data opfylder en bestemt betingelse eller ej.
\end{frame}

\begin{frame}[fragile]{Brug af betingelser i R - logiske værdier}
\protect\hypertarget{brug-af-betingelser-i-r---logiske-vuxe6rdier}{}
En lang række kommandoer og funktioner i R vil returnere \emph{logiske
værdier}.

Logiske værdier kan kun være \texttt{TRUE} eller \texttt{FALSE} (eller
missing/\texttt{NA}).

Bruger man fx disse operatorer i R, returneres altid en logisk værdi:

\begin{longtable}[]{@{}ll@{}}
\toprule()
\textbf{Operator} & \textbf{Betydning} \\
\midrule()
\endhead
\texttt{\textgreater{}} & Større end \\
\texttt{\textgreater{}=} & Større end eller lig med \\
\texttt{\textless{}} & Mindre end \\
\texttt{\textless{}=} & Mindre end eller lig med \\
\texttt{==} & Lig med \\
\texttt{!=} & Ikke lig med \\
\bottomrule()
\end{longtable}
\end{frame}

\begin{frame}{Variabelændringer}
\protect\hypertarget{variabeluxe6ndringer}{}
En væsentlig del af datahåndtering er at foretage forskellige
variabelændringer - enten i form af at rekode værdier i en variabel
eller ved at danne nye variable, som gør brug af information fra andre
variable.

\emph{Eksempler på variabelændringer:}

\begin{itemize}
\tightlist
\item
  Danne sammensatte mål (fx mål for samlet metal sundhed eller
  socioøkonomisk status)
\item
  Sammenslå kategorier (fx at behandle `meget uenig' og `lidt uenig' som
  `uenig')
\item
  Danne kategoriseringer (fx kategorier for alder, indkomst eller
  uddannelsesniveau)
\end{itemize}
\end{frame}

\begin{frame}{Variabelændringer}
\protect\hypertarget{variabeluxe6ndringer-1}{}
Gode data er meget detaljerede og nuancerede og derfor meget komplekse.

En væsentlig del af at arbejde med kvantitative metoder indebærer at
reducere kompleksitet i data på sådan en måde, at der kan foretages
meningsfulde konklusioner uden at datas integritet forværres.
\end{frame}

\begin{frame}{Rekodninger}
\protect\hypertarget{rekodninger}{}
Variabelændringer involverer ofte ændringer af værdierne i variablen.
Dette refereres til som \emph{rekodning}.

Vi kan overordnet adskille mellem tre typer af rekodning:

\begin{itemize}
\tightlist
\item
  Simpel rekodning (fx omregning af eksisterende variabel)
\item
  Ændring af kategorier (fx ved at sammenslå flere kategorier til én)
\item
  Omkodning fra numerisk til kategorisk (fx ved at lave variable for
  kategoriseringer af alder, indkomst eller andet)
\end{itemize}

Afhængig af rekodningen, kan rekodning enten være \emph{aritmetisk}
(baseret på udregning), \emph{manuel} eller baseret på
\emph{betingelser}.
\end{frame}

\begin{frame}{Hvornår laver man en ny variabel?}
\protect\hypertarget{hvornuxe5r-laver-man-en-ny-variabel}{}
Når man laver datahåndtering, bør man altid bestræbe sig efter ikke at
fjerne information fra datasættet.\\
\strut \\

\begin{columns}[T]
\begin{column}{0.5\textwidth}
\textbf{Lav ny variabel}

Værdier regnes om\\
\strut \\
Kategorier slås sammen\\
\strut \\
Rekodning fra numerisk til kategorisk\\
\strut \\
Rangorden i kategorier ændres
\end{column}

\begin{column}{0.5\textwidth}
\textbf{Behold eksisterende variabel}

Fejl rettes i variablen (fx stavefejl)\\
\strut \\
Datatype ændres (tal som tekst)\\
\strut \\
Værdier kodes til missing\\
\end{column}
\end{columns}

Er man i tvivl \emph{bør man altid lave en ny variabel}. På den måde er
der ikke information, som går tabt.
\end{frame}

\begin{frame}{Missingværdier}
\protect\hypertarget{missingvuxe6rdier}{}
Data vil ofte indeholde missing-værdier. Missing-værdier angiver
ikke-gyldige værdier; fx et manglende svar, ugyldigt svar, information
der ikke kunne skaffes eller lignende.

Missing-værdier bruges til at give en værdi uden at give en værdi
(cellerne skal indeholde noget). På den måde er man ikke nødt til at
fjerne hele rækker fra datasættet, selvom der mangler oplysninger.

Missing-værdier kan både være givet på forhånd (fx at oplysning mangler)
eller være noget, som man selv er nødt til at angive (værdier skal
behandles som missing - frasorteres analyse).
\end{frame}

\begin{frame}{Missingværdier - Kod til missing}
\protect\hypertarget{missingvuxe6rdier---kod-til-missing}{}
Der kan være flere grunde til, at man er nødt til at kode værdier til
missing:

\begin{enumerate}
\item
  Det er ikke givet at missing-værdier er kodet som missing på forhånd i
  et datasæt. Hvordan missing-værdier kodes varierer mellem programmer.
  Derfor bruger man ofte specifikke talværdier (fx 777777 eller 888888)
  til at indikere missing-værdier.
\item
  Afhængig af analysen kan man have behov for at ignorere visse
  observationer, fx for at undgå at outliers skævvrider resultatet.
\end{enumerate}

Begge situationer involverer \emph{rekodning}, hvor man erstatter de
pågældende værdier med missing (enten manuelt eller ud fra betingelser).
\end{frame}

\begin{frame}{Missingværdier - Fjern missing}
\protect\hypertarget{missingvuxe6rdier---fjern-missing}{}
Missing-værdier er ikke gyldige og kan ikke direkte indgå i beregninger.
Når det kommer til analyse, håndteres missing typisk på en af
overordnede måder:

\begin{enumerate}
\item
  Observationer med missing i relevante variable fjernes (listwise
  deletion)
\item
  Missing-værdier estimeres ud fra øvrige data (imputation)
\end{enumerate}

Vi forholder os her i dag kun til at fjerne missing.
\end{frame}

\begin{frame}{Håndtering af datatyper i R}
\protect\hypertarget{huxe5ndtering-af-datatyper-i-r}{}
R adskiller mellem objekter via deres ``class''.

Enkelte variable(/vectors) kan \emph{kun} indeholde værdier med samme
class (class sættes for vector - ikke for enkelte værdi).

Ved tvetydighed (fx variable med både tekst og tal), vil R sætte
variablen til den class, som alle værdier kan passe ind i (typisk tekst,
da tekst ikke kan konverteres direkte til tal).

Dette problem kan også opstå, hvis der gemmer sig skjulte tegn (fx
mellemrum) eller enkelte bogstaver i variablen, når datasæt indlæses.

En simpel løsning på dette er at tvinge variablen om til en anden class
(``class coercion'').
\end{frame}

\begin{frame}{Tips til god kodepraksis}
\protect\hypertarget{tips-til-god-kodepraksis}{}
God datahåndteringspraksis hænger sammen med god \emph{kodepraksis}.

God kodepraksis er overordnet et spørgsmål om at gøre koden læsbar og
forståelig \emph{i sig selv}.

Kort sagt: Hvis en anden R bruger ikke kan gennemskue hvad din kode gør,
uden at køre koden, så er det et tegn på dårlig kodepraksis.
\end{frame}

\begin{frame}[fragile]{Tips til god kodepraksis}
\protect\hypertarget{tips-til-god-kodepraksis-1}{}
\begin{enumerate}
\item
  Brug kommentarer!
\item
  Brug sigende navne til objekter (fx \texttt{\_df} for at angive at det
  er en data frame)
\item
  Behold kun kode der er relevant for at foretage analysen (fjern fx
  prints og linjer, der ikke har konsekvens for analysen)
\item
  Vær konsistent i navngivning af objekter (skift ikke mellem brug af
  \texttt{\_}, \texttt{.} og brug af versaler)
\item
  Brug mellemrum og linjeskift for at gøre koden mere overskuelig (men
  vær konsistent!)
\end{enumerate}
\end{frame}

\begin{frame}{Tips til god kod kodepraksis - Kommentarer}
\protect\hypertarget{tips-til-god-kod-kodepraksis---kommentarer}{}
God kode er kode med \emph{masser} af kommentarer!

\emph{Kommentarer kan bruges til:}

\begin{itemize}
\tightlist
\item
  Give kodeblokke overskrifter
\item
  Skrive forklaringer
\item
  Notere kritiske valg i koden (hvorfor fx et specifikt argument i
  funktion bruges)
\item
  Skrive relevante henvisninger ind
\item
  Skrive ting ind at være opmærksom på (fx hvis noget tager lang tid at
  køre)
\end{itemize}

Er du i tvivl? Skriv en kommentar!
\end{frame}



\end{document}
